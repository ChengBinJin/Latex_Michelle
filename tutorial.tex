\documentclass[11pt]{article}

%\usepackage{fullpage}
%\usepackage[top=1in, bottom=1in, left=1in, left=1in, right=1in]{geometry}
\usepackage[margin=1in, paperwidth=8.5in, paperheight=11in]{geometry}

\begin{document}

\title{Algebra 1 G/T: Chapter 1, Part 1}
\author{Mrs. Krummel}
\date{\today}
\maketitle


\section{Critical Thinking Questions (CTO's) (20 pts)}

Answer each question using complete sentences. You may choose to include sketches and/or examples to enhance your explanations. Your written answers must be typed. Sketches may be hand drawn. If you prefer not to build a complete \LaTeX \ document, I recommend using Microsoft Word's built-in equation editor or use a \LaTeX \ equation editor such as the one found at \texttt{http://www.sitmo.com/latex} as needed.

\begin{enumerate}
\item Describe a real-world situation that could be represented by the equation $a=b+3$. Include a definition for each variable.
\item Explain the difference between $-3^2$ and $(-3)^2$ in terms of order of operations.
\item Describe the relationships between natural, whole, integer, rational, irrational, and real numbers. Give one example and one non-example of each.
\item Is it always possible to find a number in between any two distinct numbers on the real number line? Explain.
	\begin{enumerate}
	\item $\left| a-b \right|=\left| b - a\right|$
	\item $\left| a+b \right| = \left| a \right| + \left| b \right|$
	\item $\left| ab \right| = \left| a \right| \cdot \left| b \right|$
	\end{enumerate}
	
\end{enumerate}

\section{Textbook Exercises (10 pts)}
\begin{itemize}
\item[] 1-1 p.6-8 \# 27, 29, 31, 33, 40
\item[] 1-2 p.12-15 \# 45, 48, 53, 54, 61, 62, 63
\item[] 1-3 p.20-23 \# 42-50, 63, 64, 65, 68-71
\item[] 1-4 p.27-30 \# 80, 82, 86, 89, 92, 94-98
\item[] 1-5 p.34-36 \# 52, 53, 54, 63, 65, 66
\item[] 1-6 p.41-44 \# 60-69, 74, 75, 88-91
\end{itemize}

\section{Additional Exercises (10 pts)}

\end{document}